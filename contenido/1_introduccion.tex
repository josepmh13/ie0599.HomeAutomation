% ----------------------
  \chapter{Introducción}
% ----------------------
\label{C:introduccion}

En este capítulo se presenta la explicación general del planteamiento del problema a resolver, justificación, metodología, antecedentes y sus respectivos objetivos y metas a alcanzar con el desarrollo de este proyecto.

%%%%%%%%%%%%%%%%%%%%%%%%%%%%%%%%
\section{Antecedentes}
%%%%%%%%%%%%%%%%%%%%%%%%%%%%%%%%

En los años recientes han surgido nuevas tecnologías que han revolucionado muchos campos y que tienden a la automatización y diseño de sistemas inteligentes para facilitar las labores cotidianas de los seres humanos, entre las nuevas áreas de desarrollo, se encuentra la domótica.

Desde el punto de vista etimológico, la palabra domótica fue inventada en Francia (país pionero en Europa) y esta formado de la contracción de "domus" (vivienda) más automática. Según \cite{Vatti2017}, la domótica implica el control y la automatización de la iluminación, calefacción, ventilación, aire acondicionado, seguridad. Eso también incluye el control y la automatización del hogar electrodomésticos como lavadoras, secadoras, hornos, refrigeradores o congeladores. La domótica es una tecnología moderna que transforma su hogar a una medida en la que puede realizar diferentes conjuntos de tareas automáticamente. Esta tecnología está constantemente actualizando su versatilidad mediante la integración de modernizado características para satisfacer las crecientes demandas de las personas, por lo que se encuentran estudios recientes del tema.

En cuanto a investigaciones previas realizadas en está área, en \cite{Hassanpour2017} se presenta el concepto de \textit{Smart Home}, el cual crea el entorno que maximiza la calidad de vida ademas del uso eficiente de los recursos energéticos y proporciona los Sistemas de Gestión de Energía (EMS). Se habla de la domótica como un componente importante en reducir el consumo de energía y el uso de energía renovable. Uno de las cuestiones fundamentales en la domótica es el costo de la automatización. De este modo, reducir el costo de la automatización es una preocupación importante en el mundo. En este articulo se le da gran importancia al costo de un sistema de automatización y desarrollo de un sistema con componentes de bajo precio.

También en \cite{Lita2017}, se presenta el diseño y el implementación de un prototipo de automatización de puertas neumática sistema destinado a ser utilizado para el control de acceso en hogares inteligentes. El sistema desarrollado en este proyecto se conecta una red local para el control del mismo. 

En el artículo presentado por \cite{Frontoni2017}, se ha desarrollado un marco que permite desarrollar rápidamente nuevos sistemas complejos de hardware y software, integran rápidamente nuevas clases de dispositivos en sistemas existentes y control y centralizar los datos. Los resultados preliminares obtenidos ya son consistente y demostrar su idoneidad y su eficacia. Este documento es importante que explica un método para desarrollar un red local, lo cual es uno lo de los objetivos que se va querer lograr alcanzar en este proyecto.

Además hay estudios de estos sistemas y sus posibles aplicaciones y ventajas, en el  artículo presentado por \cite{Errobidart2017}, se presenta una interfaz de usuario y una comunicación plataforma para un sistema modular de automatización del hogar. Su objetivo principal es contribuir a la comodidad y autónoma de usuarios que sufren algún tipo de discapacidad al usar los componentes por medio de la voz. Al mismo tiempo, propone dar una solución simple y económica a la problemática de la accesibilidad, ya que aunque hay muchos opciones de automatización en el mercado, son pocas las opciones para centralizar el control de los diferentes equipos domésticos. También en \cite{Nayyar2017}, se ofrece un proyecto que tiene como objetivo proporcionar un sistema eficiente, de bajo costo de gestión de energía para casas y que proporciona una instalación de un sistema de vigilancia de la casa. El sistema fue construido después de evaluar las características de utilidad de la vigilancia y la energía sistemas de gestión disponibles en la actualidad.

En \cite{Cabrera2016},  se describe la implementación y configuración de un asistente inteligente para controlador domótico, que permite al usuario controlar, de forma remota, el toda la casa usando comandos de voz, un control de manejo es implementado para condiciones auxiliares. Este proyecto también busca mejorar la seguridad del hogar mediante la implementación de una cámara IP para vídeo vigilancia y sensores, todo este sistema se encuentra disponible para el usuario a través de cualquier dispositivo Android \footnote{\textbf{Android}: es un sistema operativo basado en el núcleo de Linux, diseñado principalmente para dispositivos móviles.}, dando acceso a su casa inteligente desde cualquier lugar del mundo. 

A nivel nacional, en el año 2013, en el Instituto Tecnológico de Costa Rica (TEC, por sus siglas en español) estudiantes de Ingeniería en Mecatrónica, Ingeniería en Construcción, Ingeniería en Diseño Industrial, Arquitectura, Ingeniería Industrial e Ingeniería Ambiental desarrollaron una casa renovable con componentes domóticos. La vivienda estaba dotada de sensores y software que le dieron ``inteligencia" a la infraestructura; es decir, la casa podía tomar decisiones por su cuenta y sin necesidad de que los habitantes debieran ejecutar alguna acción. Además han surgido empresas que ofrecen el servicio de instalación de un sistema domótico en Costa Rica, como lo son: Home \& Office Technologies, Demsa CR, Cotisa, Almacenes Mauro, entre otros; sin embargo muchos de empresas se limitan a la instalación de luces de encendido automático.

Desde el punto de vista comercial, en el área de la construcción muchas de las tendencias en países desarrollados es con casas con sistemas domóticos con tecnología avanzada y muchas de las viviendas que se construyen en los años recientes, se diseñan haciendo uso de este tipo de tecnología, por lo que ha tomado fuerza el término de ``arquitectura domótica". Además ha sido tendencia reciente en el mercado hotelero, pudiendo gestionar de forma eficiente el consumo energético, a la vez que optimizar y aprovechar máximo los recursos para conseguir la máxima rentabilidad; todo ello con una inversión mínima. Entre las principales razones por las que se encuentra en tendencia es porque: 

\begin{itemize}
\item Optimización de tiempos
\item Reduce riesgos de seguridad
\item Aumenta el entretenimiento
\item Potencia estilos de vida
\end{itemize}

La domótica además ofrece ventajas monetarias, según la Asociación Española de Domótica e Inmótica (CEDOM, por sus siglas en español), esta tecnología ayuda a mejora de los sistemas de motorización del consumo energético en hogares y edificios, lo que posibilitaría una reducción del gasto de energía del 30\%. También permite actuaciones autónomas en caso de incendio,escapes de agua o gas e incrementa la seguridad de nuestra vivienda.  

En conclusión, como se menciona en \cite{Sechi2010}, la domótica se esta convirtiendo en un campo no tan futurista, de hecho, la cantidad de este tipo de sistemas en el mercado esta creciendo rápidamente. La instalación de estos sistemas permiten mejorar la comodidad y la seguridad de una casa a través de la integración de los conceptos tradicionalmente asociado con el ambiente domestico con tecnologías de nueva generación, por lo que la investigación y comercialización es está área esta en tendencia y constante crecimiento.

%%%%%%%%%%%%%%%%%%%%%%%%%%%%%%%%
\section{Justificación}
%%%%%%%%%%%%%%%%%%%%%%%%%%%%%%%%

En la actualidad, uno de los principales problemas que se encuentra con la implementación de los sistemas domóticos, es que son instalados en los hogares por empresas privadas que asocian el producto con el servicio de monitoreo y control de los dispositivos conectados al sistema domótico a través del uso de un servidor remoto. Por lo tanto en este proyecto se va a realizar haciendo uso de una red local, entre las principales ventajas de hacer uso de este tipo de red son:

\begin{itemize}
\item Se puede centralizar la gestión de los usuarios y las contraseñas.
\item Logra minimizar el número de credenciales dentro de la red.
\item El costo monetario es bastante económico.
\item Permite administrar la seguridad de los mismos con algoritmos cifrados.
\item Es escalable, por lo que se puede cambiar el tamaño del servidor local fácilmente en el caso de ser necesario.
\end{itemize}

Otro inconveniente de estos sistemas es el precio y la complejidad. Sin embargo, la tecnología \textit{Sonoff} es un dispositivo de la empresa \textit{Itead} muy económico y sencillo de usar. Por lo tanto, en este proyecto se va a hacer uso de  estos componentes para desarrollar un sistema domótico. En la investigación realizada, actualmente no sean realizado proyectos similares con este tipo de componentes de marca comercial y además una de las principales ventajas es que estos dispositivos se puede manejar con ingeniería reversa, por lo que se pueden editar configuraciones de fábrica de los componentes para lograr hacer que los dispositivos trabajen en una red local con funciones programadas para realizar operaciones específicas adaptadas a las capacidades de hardware del componente, e incluso el fabricante da acceso a documentación técnica de fabricación y de como programarlo. Por lo tanto, en este proyecto se propone usar esta marca al ser un dispositivo totalmente configurable, que se adapta a las necesidades del usuario y por el bajo costo. 

Otro aspecto a considerar es que se pretende la implementación física del sistema dómotico y del diseño de la interfaz gráfica, así como la utilización de diversos componentes como cámaras, sensores, interruptores, entre otros; que son típicos de este tipo de sistemas domóticos. Para el desarrollo de la interfaz gráfica se va usar \textit{QT5}, ya que es una biblioteca desarrollada como un software libre y de código abierto, además que en la actualidad es una de las herramientas más usadas para el desarrollo de interfaces gráficas.

Además, se pretende realizar un estudio de que como se implementan este tipo de sistemas y se estudiarán posibles usos de lo mismo como lo puede ser en sistemas de seguridad privada con el uso de cámaras, ahorro y monitoreo del consumo de energía, accesibilidad a personas con discapacidades, ya que facilita el control de los componentes electrónicos del hogar, entre otros.


%%%%%%%%%%%%%%%%%%%%%%%%%%%%%%%%
\section{Planteamiento del problema}
%%%%%%%%%%%%%%%%%%%%%%%%%%%%%%%%

En general, en los sistemas domóticos se trata lograr la integración de componentes electrónicos en las viviendas para brindar servicios de gestión energética, seguridad, bienestar y comunicación.

Empresas comerciales que ofrecen el servicio de instalación de sistemas domóticos en los hogares hacen de servidores remotos bajo dominio de la empresa que instala el sistema, lo que para el usuario puede tener implicaciones de privacidad y seguridad importantes.

En cuanto a los componentes electrónicos de sistemas domóticos actuales, se esta acostumbrado a trabajar con cajas negras por ejemplo, por lo que la configuración por fábrica no es sencillo modificarlo. Básicamente porque no tenemos acceso a la documentación técnica como esquemas eléctricos o el código, lo que imposibilita su modificación. Sin embargo, las tecnologías libres nos ayudan en este sentido, compartiendo el conocimiento con la comunidad. Esto nos brinda muchas posibilidades, ya que, nos ayuda a adaptar los dispositivos a nuestros requerimientos y no al contrario. Este es el caso de los componentes Sonoff, que aplica ingeniería inversa en los productos que ofrece. 

Otro aspecto relevante es el costo monetario de los componentes electrónicos usados en estos sistemas y el precio de instalación de empresas que ofrecen este servicio es elevado, lo que imposibilita que este tipo de sistemas se haya logrado comercializar completamente hasta el momento, por lo que el mercado apenas se encuentra en crecimiento.

Por lo tanto, el problema a resolver implica la creación de una red local el cuál puede ser usada para poder conectar componentes electrónicos Sonoff que son de bajo costo y que permiten ingeniería inversa, además de un control de los componentes por medio de una interfaz amigable con los usuarios y funcional.

%%%%%%%%%%%%%%%%%%%%%%%%%%%%%%%%
\section{Objetivos}
%%%%%%%%%%%%%%%%%%%%%%%%%%%%%%%%
A continuación se presenta el objetivo general y específicos que se quieren alcanzar con el desarrollo de este proyecto:

\subsection*{Objetivo General}
\begin{itemize}
\item Diseñar e implementar un sistema de domótico en una red local haciendo de componentes electrónicos de bajo costo comerciales y sistemas embebidos con una interfaz gráfica para su respectivo control.
\end{itemize}
\subsection*{Objetivos Específicos} 
\begin{itemize}
\item Crear una red local que ayuden a la comunicación de componentes electrónicos conectados a la red que serán usados en el sistema domótico.
\item Programar los componentes electrónicos Sonoff para la implementación y utilización en un sistema domótico.
\item Diseñar una interfaz gráfica que se capaz de controlar los dispositivos del sistema creado.
\item Validar el correcto funcionamiento del diseño del sistemas domótico realizado con la creación de una maqueta a escala.
\end{itemize}

%%%%%%%%%%%%%%%%%%%%%%%%%%%%%%%%
\section{Alcances del proyecto}
%%%%%%%%%%%%%%%%%%%%%%%%%%%%%%%%

En este proyecto se propone realizar el diseño e implementación de un sistema dómotico haciendo uso de componentes comerciales de la marca Sonoff, para que se conecten a un servidor local el cual va poder ser manipulado por medio de una interfaz gráfica que se desarrollará haciendo uso de una biblioteca de software libre llamada \textit{QT5}. Además para efectos prácticos, el proyecto será validado en una maqueta a escala de una casa.

El proyecto tendrá impacto al usar un servidor local y no al hacer uso de un servidor remoto, como las que se ofrecen actualmente en las compañías comerciales que ofrecen la instalación de estos sistemas, lo cuál tienen impactos de seguridad, confiabilidad, almacenamiento, entre otros. Además de usar componentes emergentes en el mercado que ofrecen la ventaja de poder aplicar ingeniería inversa, lo que los hace totalmente programables a realizar las labores que se desee y a su vez, el costo monetario de dichos componentes es relativamente bajo a la competencia de productos similares. Con el desarrollo de la interfaz gráfica se buscará facilitar el uso de los componentes del sistema domótico, creando una forma amigable y sencilla de manipulación de los componentes para un posible usuario del sistema diseñado.  

Por otro lado, se tiene planeado validar el proyecto en una maqueta a escala de una casa, ya que, por limitaciones físicas, económicas y de seguridad es imposible la implementación del sistema en una casa real. Por lo tanto al usar maqueta se puede validar el funcionamiento, además se puede trabajar, desarrollar y ejemplificar el uso del sistema haciendo uso de un prototipo.   

En este proyecto, no se tomará en cuenta un análisis de costos y posibles beneficios económicos de la comercialización del producto final a desarrollar, pero una vez finalizado el proyecto, se pretende que pueda ser tomado en cuenta para participar en incubadoras de empresas que ofrece la Universidad de Costa Rica (UCR, por sus siglas en español) para proyectos de índole similar, como lo son Auge UCR (\url{http://www.augeucr.com/es}), pero cabe aclarar que está parte no será tomado en cuenta como parte del proyecto.

%%%%%%%%%%%%%%%%%%%%%%%%%%%%%%%%
\section{Metodología}
%%%%%%%%%%%%%%%%%%%%%%%%%%%%%%%%
Para el desarrollo del proyecto, se propone dividir el mismo en secciones que faciliten la planificación del desarrollo de las tareas del proyecto, dichas etapas se presentan a continuación:

%-----------------------------------
\subsection{Etapa de desarrollo del servidor local}
%-----------------------------------
Primeramente, se investigará los procedimientos a seguir para creación de una red local. Implementar servidor GNU/Linux para que sirva como servidor de almacenamiento y de conexión para los dispositivos que se vayan a conectar a la red.

%-----------------------------------
\subsection{Etapa de integración de los componentes electrónicos tipo Sonoff}
%-----------------------------------
En esta etapa, se investigará el proceso de configuración, uso e instalación de los componentes comerciales tipo Sonoff. Además se debe de programar en lenguaje de programación C/C++ los componentes para realizar labores específicas del sistema automatizado e que sean capaces de conectarse a la red local.

%-----------------------------------
\subsection{Etapa de diseño de la interfaz gráfica}
%-----------------------------------
Se procederá a diseñar una interfaz gráfica que facilite el uso del sistema realizado a un posible usuario del sistemas domótico. Para esto se pretende usar software libre como lo es QT5, que es uno de los sistemas más comúnmente usado en la industria para el diseño de interfaces y que se puede programar en C/C++. Por lo tanto, primeramente se realizará un proceso de estudio para la adaptación del diseño de interfaces con este software y posteriormente se creará un sistema que sea capaz de adaptar los componentes electrónicos y la red local para que sea posible manejar los componentes de manera sencilla y según las capacidades de hardware de los componentes.

%-----------------------------------
\subsection{Etapa de validación}
%-----------------------------------
Por último, en está etapa se va a validar el sistemas domótico, ya que, por limitaciones físicas y económicas no es viable probarlo en una casa real, se procederá a desarrollar una maqueta en la que se va a montar el sistema de manera que puedan ser instalados todos los componentes que serán usados en el sistema. Una vez montado el sistema en la maqueta, se harán pruebas de funcionalidad para verificar el funcionamiento de la red local, los componentes electrónicos instalados y la interfaz gráfica. 

%%%%%%%%%%%%%%%%%%%%%%%%%%%%%%%%
\section{Procedimiento de evaluación}
%%%%%%%%%%%%%%%%%%%%%%%%%%%%%%%%
Al finalizar el presente proyecto, se generará un estudio en el cual se detallarán los resultados obtenidos, en el que se espera describir los procedimientos a seguir para el desarrollo de un sistema domótico con tecnología de bajo costo y se obtendrán las conclusiones de las posibles ventajas y desventajas de estos sistemas. 

A continuación, se describe los resultados que se esperan obtener en el desarrollo de cada una de las etapas de este proyecto:

%-----------------------------------
\subsection{Etapa de desarrollo del servidor local}
%-----------------------------------
En esta etapa del proyecto, se debe obtener un servidor local funcionala que este corriendo bajo una computadora o un sistema embebido como lo puede ser una Raspberry Pi.

%-----------------------------------
\subsection{Etapa de integración de los componentes electrónicos tipo Sonoff}
%-----------------------------------
Los componentes electrónicos Sonoff, deben estar estar programados para que estén conectados a las red local y funcionando de manera adecuada según sus respectivas cualidades de hardware.

%-----------------------------------
\subsection{Etapa de diseño de la interfaz gráfica}
%-----------------------------------
Se debe de obtener una interfaz gráfica en QT5 con un diseño amigable y funcional y capaz de conectarse a la red local para controlar los componentes electrónicos conectados a la red domótica.

%-----------------------------------
\subsection{Etapa de validación}
%-----------------------------------
Se espera obtener una maqueta con cada uno de los componentes disponibles y programados, con el servido local funcionando y que pueda ser controlador con la interfaz gráfica desarrollada.

%%%%%%%%%%%%%%%%%%%%%%%%%%%%%%%%
%\section{Estructura del trabajo}
%%%%%%%%%%%%%%%%%%%%%%%%%%%%%%%%
%PENDIENTE...

