% LA NOMENCLATURA
% ---------------

% La nomenclatura se realiza con el paquete 'nomencl'. Para ingresar un nuevo elemento, se debe usar el comando \nomenclature{símbolo}{definición}, ya sea en este archivo nomenclatura.tex (más fácil para encontrar y editar), o en cualquier parte del documento (probablemente cuando se introduce una nueva variable o constante). Para más opciones del paquete, favor referirse a su documentación (https://www.ctan.org/pkg/nomencl). También hay una buena guía de uso en https://www.sharelatex.com/learn/Nomenclatures.

% Formato recomendado
% -------------------

% Variable o constante matemática
% \nomenclature{$V$}{Tensión eléctrica}

% Acrónimo
% \nomenclature{TBH}{Para ser honesto (del inglés \textit{To Be Honest})}

% Si únicamente existen acrónimos del inglés, se puede omitir la frase 'del inglés'. La definición no tiene punto al final.
\nomenclature{$IoT$}{Internet de las Cosas (del inglés \textit{Internet of Things})}
\nomenclature{$HMI$}{Interfaz de Usuario (del inglés \textit{Human-Machine Interface})}
\nomenclature{$EMS$}{Sistemas de Gestión de Energía (del inglés \textit{Energy Management System})}
\nomenclature{$CLI$}{Interfaz de Línea de Comandos (del inglés \textit{Command Line Interface})}
\nomenclature{$NUI$}{Interfaz Natural de Usuario (del inglés \textit{Natural User Interface})}
\nomenclature{$AC$}{Corriente Alterna}
\nomenclature{$PID$}{Controlador Proporcional Integral Derivativo}
\nomenclature{$DC$}{Corriente Directa}
\nomenclature{$V$}{Volt}
\nomenclature{$A$}{Amperio}
\nomenclature{$W$}{Watt}
\nomenclature{$i$}{Corriente Eléctrica}
\nomenclature{$PWM$}{Modulación por Ancho de Pulso (del inglés \textit{Pulse Width Modulation})}
\nomenclature{$GUI$}{ Interfaz Gráfica de Usuario (del inglés \textit{Graphical User Interface})}
\nomenclature{$EIE$}{Escuela de Ingeniería Eléctrica de la Universidad de Costa Rica}
\nomenclature{$UCR$}{Universidad de Costa Rica}
\nomenclature{$TEC$}{Instituto Tecnológico de Costa Rica}
\printnomenclature